\section{Response to Reviewer \#1 (Akib Shahriyar)}
% \subsection*{Overall Comments}
% \begin{mdframed}
% \begin{quote}
% Place the overall comments of Reviewer \# 1 here.
% \end{quote}
% \end{mdframed}

% \subsection{Response} 
% We appreciate your careful review and detailed feedback. Our focus in the revised manuscript was to clearly state the novelties and contributions. We hope that you find the following response satisfactory.



%%%%%%%%%%%%%%%%%%%%%%%%%%%%%%%%%%%%%%%%%%%%%%%%%%%%%%%%%%%%%%%%%%%%%
\subsection{Reviewer Comment}
\begin{mdframed}
\begin{quote}
The authors have developed a stemming algorithm and tested it on a relatively small dataset (100 pairs of Nepali news articles).
\end{quote}
\end{mdframed}

\subsection{Response} 
We have introduced a new dataset called NEP-PLAG2019v1, consisting of 10,000
pairs of articles. Please refer to the 'Introduction to new publicly available dataset'
point in the major changes summary.\\

\noindent\rule{17cm}{2.0pt}

%%%%%%%%%%%%%%%%%%%%%%%%%%%%%%%%%%%%%%%%%%%%%%%%%%%%%%%%%%%%%%%%%%%%%%
\subsection{Reviewer Comment}

\begin{mdframed}
\begin{quote}
The authors have claimed that no prior work has been done on the pre-processing
of Devanagari scripts without any substantial literary reference. The claimed
contributions are not completely novel. Previous work has already been done on
developing a stemming algorithm for Devanagari Scripts. The experimental results
are presented without comparing them with any similar method/prior work.
\end{quote}
\end{mdframed}

\subsection{Response} 
We have realized the mistake pointed by the reviewer in regards to our claim
that no prior work has been done. We have added references to prior works on the
preprocessing of Devanagari scripts in the Introduction section. Please refer to
the 'Improvement in the Introductions section' point in the major changes
summary.

New experiment results have been added to compare them with prior works in the
C. Discussions and Comparisons sub-section under VI. Results and Discussions
section.\\

\noindent\rule{17cm}{2.0pt}

%%%%%%%%%%%%%%%%%%%%%%%%%%%%%%%%%%%%%%%%%%%%%%%%%%%%%%%%%%%%%%%%%%%%%%
\subsection{Reviewer Comment}

\begin{mdframed}
\begin{quote}
The explanation of the methods is insufficient for a reader to replicate it
thoroughly. Specifically, the dataset/corpora are not publicly available.
Moreover, authors have manually annotated the dataset and also collected
stop-words from various online locations. There are no references in the paper
that could lead any reader to those resources. These points should be taken care
of by the authors.
\end{quote}
\end{mdframed}

\subsection{Response} 
Further explanations of the methods have been added by creating a more
sub-sections. The Methodology section in the original manuscript has been
replaced by steps of the methods so that they can be explained in detail. 

The plagiarism dataset, the code and all the used corpora including stop-words have been
provided publicly with links included in the footnotes so that the methods can
be replicated by the readers easily.\\


\noindent\rule{17cm}{2.0pt}

%%%%%%%%%%%%%%%%%%%%%%%%%%%%%%%%%%%%%%%%%%%%%%%%%%%%%%%%%%%%%%%%%%%%%%
\subsection{Reviewer Comment}

\begin{mdframed}
\begin{quote}
The Methodology Subsection and Stemming and Lemmatization algorithm subsection need a complete overhaul. The text size in the figures is too small for the usual reading. They should be updated. Also, there are some typographical errors and it needs proofreading.
\end{quote}
\end{mdframed}

\subsection{Response} 
Please refer to the previous response regarding the sub-division of methodology
section. The figures have been updated to increase visibility and the large figure
containing the entire algorithm flow diagram has been placed in a separate one
column page.

All the typographical errors have been checked and corrected.\\

\noindent\rule{17cm}{6.0pt}
