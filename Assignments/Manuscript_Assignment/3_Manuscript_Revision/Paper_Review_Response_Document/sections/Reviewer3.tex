\section{Response to Reviewer \#3 (Dingrong Wang)}
% \subsection*{Overall Comments}
% \begin{mdframed}
% \begin{quote}
% 	 Reviewer \# 3 - Overall Comments here
% \end{quote}
% \end{mdframed}

% \subsection{Response} 
% We would like to thank you for you positive feedback. Your detailed comments have considerably helped with improving the clarity of the revised manuscript.

% \noindent\rule{17cm}{2.0pt}

%%%%%%%%%%%%%%%%%%%%%%%%%%%%%%%%%%%%%%%%%%%%%%%%%%%%%%%%%%%%%%%%%%%%%%
\subsection{Reviewer Comment}
\begin{mdframed}
\begin{quote}
The work is very solid and impressive. But I think in order for this paper to be
published, you still need some revision. For example, you should explain more
detailly about how to do lemmatization, such as why there still exists a check
for suffix and prefix after previous two check about prefix and suffix?
\end{quote}
\end{mdframed}

\subsection{Response} 
Further explanations of the methods have been added by creating a more
sub-sections. The Methodology section in the original manuscript has been
replaced by steps of the methods so that they can be explained in detail. 

The specific questions regarding the check for suffix and prefix after previous
two checks have been addressed in the algorithm description in point 6 of the 
sub-section E. Stemming and Lemmatization Algorithm. Likewise, the code 
and datasets have been made publicly available
so that the methods can
be replicated by the readers easily.\\

\noindent\rule{17cm}{2.0pt}

%%%%%%%%%%%%%%%%%%%%%%%%%%%%%%%%%%%%%%%%%%%%%%%%%%%%%%%%%%%%%%%%%%%%%%
\subsection{Reviewer Comment}
\begin{mdframed}
\begin{quote}
	And another question, why do you choose to recombine suffix, but not prefix,
	is there a combination problem here? Besides, the Jaccard similarity you talk
	about in the end just concerns the vocabulary info, not the frequency info,
	right? I think you should explain more about these questions for the reader to
	get a better understanding.
\end{quote}
\end{mdframed}

\subsection{Response}  
Please refer to the previous response regarding the addressing of questions in
the algorithm description. This question has also been addressed now. 

The issues with Jaccard similarity and the reason for discarding it for
classification of texts has been added at the end of the B. Experiment
sub-section of VI. Results and Discussions section. \\

\noindent\rule{17cm}{2.0pt}

%%%%%%%%%%%%%%%%%%%%%%%%%%%%%%%%%%%%%%%%%%%%%%%%%%%%%%%%%%%%%%%%%%%%%%

\subsection{Reviewer Comment}
\begin{mdframed}
\begin{quote}
And if you want to improve your result, I think you should try some deep
learning method in NLP to dig out more semantic and contextual information to
construct feature vector and further compute the similarity extent.
\end{quote}
\end{mdframed}

\subsection{Response} 
We agree that there are several deep learning methods which can provide better
results for plagiarism detection. We have added these methods as part of the
limitations and future work in the VII. Conclusion section. The current approach
has produced fair results for the dataset being used, hence we have retained the
main approach in the paper. However, we intend to use deep learning approaches
in future for increasing the performance accuracy in more complex dataset.\\


\noindent\rule{17cm}{6.0pt}
